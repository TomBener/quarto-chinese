\usepackage[zihao=-4,linespread=1.5]{ctex}

\usepackage{newpxtext}
% \setmainfont{Palatino}
\setCJKmainfont[AutoFakeBold=5,  % 伪粗体
    ItalicFont=KaiTi,  % 中易楷体
    ]{FangSong}  % 中易仿宋
    % ]{SimSun} % 华文宋体

% Define HUGE font size for the title
\makeatletter
\newcommand\HUGE{\@setfontsize\Huge{36}{44}}
\makeatother

% Change page label for cover page
\newcommand{\CoverName}{Cover}

% Add space between Chinese characters
\newcommand{\cnspace}[2]{\ziju{#1}#2}

% Redefine the font size of the quote environment
\let\oldquote\quote\renewcommand\quote{\oldquote\zihao{5}}

% TOC style
\usepackage[titles]{tocloft}
\renewcommand{\cftsecleader}{\cftdotfill{\cftdotsep}}
% \setlength{\cftbeforesecskip}{5pt}

% Footer style
\usepackage{fancyhdr}
\fancypagestyle{plain}{%
\fancyhf{} % clear all header and footer fields
\fancyfoot[C]{\footnotesize\selectfont\thepage}
\renewcommand{\headrulewidth}{0pt}
\renewcommand{\footrulewidth}{0pt}}
\pagestyle{plain}
\setlength{\footskip}{1.5cm}

% Define the color of hyperlinks
\definecolor{urlblue}{rgb}{0.19,0.54,0.92}
\definecolor{winered}{rgb}{0.5,0,0}
\definecolor{wordblue}{RGB}{40,85,150}

% Define replacements for Chinese quotation marks
\usepackage{newunicodechar}
\newunicodechar{«}{“}
\newunicodechar{»}{”}
\newunicodechar{‹}{‘}
\newunicodechar{›}{’}

% Avoid overflow of the code block
\usepackage{fvextra}
\DefineVerbatimEnvironment{Highlighting}{Verbatim}{breaklines,commandchars=\\\{\}}

\usepackage{etoolbox}
% 参考文献样式
\AtBeginEnvironment{CSLReferences}{%
    % \zihao{5}  % 字号;默认继承正文字号(\zihao{-4})
    % \linespread{1.2}\selectfont  % 行距;默认 1.0
    % \setlength{\itemsep}{0.5\baselineskip}  % 条目间距;默认 #2\baselineskip
    % \setlength{\parskip}{0pt}  % 段间距;默认 0pt
    \setlength{\cslhangindent}{2.65em}  % 悬挂缩进距离;默认 1.5em
    \normalspacedchars{[]}  % 移除文献标识符中括号 [ 前与汉字间的空格,默认 CJK 会自动插入空格
}
